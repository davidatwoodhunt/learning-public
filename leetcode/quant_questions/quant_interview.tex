\documentclass[a4paper,10pt]{article}
% Preamble
\usepackage{latexsym}
\usepackage[utf8]{inputenc}
\usepackage{geometry}
\usepackage{titlesec}
\usepackage{enumitem}
\usepackage{hyperref}
\usepackage{darkmode}
\usepackage{listings}
\usepackage{xcolor}
\usepackage{pgfplots}
\pgfplotsset{compat=1.17}
\enabledarkmode

\definecolor{lilac}{HTML}{C8A2C8}
\lstset{
    basicstyle=\ttfamily\footnotesize,   % Code font and size
    keywordstyle=\color{lilac},          % Keywords color
    commentstyle=\color{gray},           % Comments color
    stringstyle=\color{pink},            % Strings color
    numbers=left,                        % Line numbers on the left
    numberstyle=\tiny\color{gray},       % Line number style
    stepnumber=1,                        % Line number increment
    frame=single,                        % Adds a frame around the code
    breaklines=true,                     % Line breaking
    tabsize=4                            % Tab size
}



\title{Questions for Quant Interviews} % Title of the document
\author{David Hunt} % Author(s) of the document

\begin{document}

% Title and author
\maketitle

% Questions
\section*{Questions}

\textbf{Question 1 (statistics)}: How do you test whether a data sample is normal or not?
\vspace{0.25cm} 
\noindent
\begin{sloppypar}
        There are several ways to test for normality. Most easily one can run a z or t test given the number of samples. 
        This will show if the data is normally distributed. Another way is to use a Q-Q plot to compare the data to a normal distribution.
        Also to look at the skewness and kurtosis of the data and compare to the moments of the normal distribution.
\end{sloppypar}
\vspace{0.5cm} 
\noindent
\textbf{Question 2 (math)}: Show that a set is convex if and only if its intersection with any line is convex. Show that a set is affine if and only if its intersection with any line is affine.
\vspace{0.25cm} 
\noindent
\begin{sloppypar}
        By the definition of a conic set, a set is convex if for any two points in the set, the line conecting them is wholly within the set.
        the proof should follow that if you pick any two arbitrary points in the set $x$ and $y$ and draw a line between them, the line will be wholly within the set.
        the proof for affine is similar, but the line between the two points is not wholly within the set.
\end{sloppypar}
\vspace{0.5cm} 
\noindent
\textbf{Question 3 (finance)}: explain the discount rate and why its important 
\vspace{0.25cm}
\noindent
\begin{sloppypar}
        The discount rate is the rate at which future cash flows are discounted to the present value. in the case of US bonds
        it can be thought of as the risk free rate from the US treasury. In this case we can think of rates on a curve.
        If i'm going to be lending money for $n$ amount of time then I will need to discount the cash flows of the interest rate i receive by what i could recieve otherwise in the market.
        This is important because it allows us to compare the value of money at different times.
        in the case of lending for 30 years, i should use the rate that i \textit{expect} the 30yr rate to be, in this case, the yield on 30yr treasury bonds. 
\end{sloppypar}
\vspace{0.5cm} 
\noindent
\textbf{Question 4 (probability)}: Monty Hall problem
\vspace{0.25cm}
\noindent
\begin{sloppypar}
        The Monty Hall problem is a probability puzzle based on the game show "Let's Make a Deal". 
        In this case, you should always switch doors. this can be explained as follows:
        The probability that you pick the "wrong" door at the start is $\frac{2}{3}$, 
        and the probability that you pick the "right" door is $\frac{1}{3}$.
        if you initially picked the door with the car you will lose if you switch;
        but that only happens $\frac{1}{3}$ of the time.
\end {sloppypar}
\vspace{0.5cm}
\noindent
\textbf{Question 5 (stats)}: Random variable X is distributed as N(a,b) and random variable Y is distributed as N(c,d). What is the distribution of X+Y, X-Y, X*Y, and X/Y?
\vspace{0.25cm}
\noindent
\begin{sloppypar}
    \begin{itemize}
        \item{X+Y: $N(a+c, \sqrt{b^2+d^2})$}: means are additive but std is root of variance added
        \item {X-Y: $N(a-c, \sqrt{b^2+d^2})$}: means are subtracted but std is root of variance added. variance represents uncertaintly which is always +ve
        \item {X*Y}: is not normally distributed 
        \item {X/Y}: is not normally distributed
    \end{itemize}
    
    For the case of X*Y the mean will look like $E[X] * E[Y] = a \times c$
    However, the variance will be 
    $$E[X^2]E[Y^2] - E[X]E[Y]^2 = b^2d^2 + a^2c^2 - a^2c^2 = b^2d^2$$
    for X/Y: this is not normally distributed and skewed with heavy tails. 
            Ther is no closed form solution for this distribution but can be approximated with the log-normal distribution.

\end{sloppypar}
\vspace{0.5cm}
\noindent
\textbf{Question 6 (options)}: what is put-call parity / what are teh assumptions and risks in trading
\vspace{0.25cm}
\noindent
\begin{sloppypar}
    The underlying logic is that the value of a call option and a put option with the same strike price and expiration date should be equal.
    This is because if you buy a call and sell a put, you have the same exposure as buying the stock.
    The put-call parity formula is $C - P = S - K/(1+r)^t$ where $C$ is the call price, $P$ is the put price, $S$ is the stock price, $K$ is the strike price, $r$ is the risk-free rate, and $t$ is the time to expiration.

    the assumptions you have are that these are European options not American, so no early exercise. 
    Also, you're assuming a constant risk-free rate and no dividends.
\end{sloppypar}
\vspace{0.5cm}
\noindent
\textbf{Question 7 (programming)}: is this valid C++ code /  what does it print?
\begin{lstlisting}[language=C++]
    cout << (int *) "home of the jolly bytes";
\end{lstlisting}
\vspace{0.25cm}
\noindent
\begin{sloppypar}
    This will print the pointer address of the string "home of the jolly bytes".
    The code is valid but the output is probably not what the programmer intended
\end{sloppypar}
\vspace{0.5cm}
\noindent
\textbf{Question 8 (econometrics)}: is an AR(p) process stationary? why or why not? 
\vspace{0.25cm}
\noindent
\begin{sloppypar}
    It can be. It is an autoregressive process. Not all AR processes are stationary such as an AR(1) process with a coefficient of 1.
    For an AR(p) process to be stationary, the roots of the characteristic equation must be outside the unit circle.
    This is because the AR process is a linear combination of past values of the series.
    If the roots are inside the unit circle, the series will explode to infinity.
    this is defined by the equation:
    $$X_t = \phi_1X_{t-1} + \phi_2X_{t-2} + ... + \phi_pX_{t-p} + \epsilon_t$$
    where $\epsilon_t$ is white noise.
    $$1 - \phi_1z - \phi_2z^2 - ... - \phi_pz^p = 0$$
\end{sloppypar}
\vspace{0.5cm}
\noindent
\textbf{Question 9 (econometrics)}: what is the difference between AR and MA processes?
\vspace{0.25cm}
\noindent
\begin{sloppypar}
    \noindent
    AR(p) process is the linear combination of its own past values. Which means it exhibits self dependence over time. \\ 
    AR(p) values decay gradually depending on the value of the coefficients $\phi_n$ in the model.\\ 
    MA(q) process depends on \textbf{past random shocks} that move the average value of the series. \\ 
    MA(q) values depend on the parameter Q and once this value is aged enough it drops out of the model.\\
    AR(p) can be stationary if the roots of the characteristic equation are outside the unit circle.\\
    MA(q) is always stationary since its a linear combination of past shocks.\\
\end{sloppypar}
\vspace{0.5cm}
\noindent
\textbf{Question 10 (econometrics)}:  How do you determine the order of an AR or MA process?
\vspace{0.25cm}
\noindent
\begin{sloppypar}
    \noindent
    You can use the ACF(\textit{Autocorrelation Function}) and PACF(\textit{Partial Autocorrelation Function}) to determine the order of an AR or MA process.\\ 
    \\
    \textbf{AR(P)}:
        \indent
        \begin{itemize}
            \item{ACF}: will decay gradually overtime
            \item{PACF}: will have a sharp cutoff after lag P
            \item for example if you have a sharp cutoff after P=3 then you have an AR(3) process
        \end{itemize}
    \textbf{MA(Q)}:
        \indent
        \begin{itemize}
            \item{ACF}: cutoff sharply after lag Q
            \item {PACF}: will decay gradually overtime
            \item for example if you have a sharp cutoff after Q=3 then you have an MA(3) process
        \end{itemize}
    
\end{sloppypar}
\vspace{0.5cm}
\noindent
\textbf{Question 11 (hedging)}: How do you determine the optimal hedge for 100mm short CDS position?
\vspace{0.25cm}
\noindent
\begin{sloppypar}
    \noindent
    \indent
    \begin{itemize}
        \item{Assume no recovery}
        \item{5yr CDS on stock ABC you collect 200bps on the contract}
        \item{in default, perfect correlation between equity and credit} 
        \item{How do you hedge / what are the risks there}
    \end{itemize}

    By selling CDS you are long the credit risk of the company, so if they default you get allocated the negative pnl. 
    To hedge this, all you care about is the terminal state. It's 5 yr CS so you need to think about the 5yr state of the credit.
    Since in default there is nearly perfect correlation between equity and credit we can assume the default case is a equity position efficively. 
    The best way to hedge out tail risk on equity is through crash options. we can buy some crash puts on the stock for 5yr term and at a low strike. 
    We would want to have a low strike so that we're not over paying for the optionality of the not crash but degredation case. \\
    \\
    To price the correct amount for the option we would want the 10\% or 5\% away from zero strike. ideally here the option is probably going to be trading for
    10\% or 5\% of the amount we're receiving on our CDS spread over the 5yr term. So here we're recieving efficively \$10mm of CDS insurance payment over the 5 yr term. 
    We want to pay less than the premium we're receiving for the CDS. The put is going to be trading at \textit{near} zero.
    But 5yr puts are probably going to be OTC so you're subject to bank vagaries. (ignore for now)
    puts are probably trading at ~~\$1 (in the 10 case) since stock is trading at 100(assume) and in the case of default which is implied at 2\% /year for the next 5 years,
    we can think of the option as equating to the 2\% default rate. so the option is trading at 2\% * 5 of the stock price with a 100 ratio.
    \\

    Risk we're running here is that we're paying up front for the option but we're receiving cash flows over time. so discounting is important.
    In addition, if the company defaults tomorrow we don't get our cash flow payments, though we do get our option expiry. so premium isn't worthless and we're probably net up 
    though we have overpaid for $\theta$. 

\end{sloppypar}
\vspace{0.5cm}
\noindent
\textbf{Question 12 (Finance)}: What is the intuition behind CAPM?
\vspace{0.25cm}
\noindent
\begin{sloppypar}
    CAPM or Capital Asset Pricing Model, is a way to value capital assets in terms of their risk adjusted return.
    A way to think about it is, it is attempting to put a market value on the intrinsic use of a capital asset and its depreciation over time. 
    $$ E(R_i) = R_f + \beta_{i} * (R_m - R_f)$$

\end{sloppypar}
\vspace{.5cm}
\noindent
\textbf{Question 13 (econometrics)}: When modeling binary-choice problems, what are the advantages of using logic over probit?
\vspace{.25cm}
\noindent
\begin{sloppypar}
    \noindent
    \textbf{Logit}:
    $$ logit(x) = \frac{1}{1+e^{-x}}$$
    \textbf{Probit}:
    $$ probit(x) = \Phi(x) $$

    logit is the sigmoid function which uses a logistic distribution of errors, and probit uses a normal distribution. 
    for binary choice problems I would prefer to use the Logit function as it has fatter tails, better handling a 0 / 1 case. 
    Functionally both will work, however Logit is computationally simpler and handles the tails well
\end{sloppypar}


\textbf{Question 13 (QR)}: Given a series of returns in sample and out of sample.along with a set of alphas and risk factors. How can you set a positions column to maximize sharpe?
\vspace{.25cm}
\noindent
\begin{sloppypar}
    Assume that returns are modeled as 
    $$ r = Xf + Z \beta  + \mu $$
    Where f is the risk facors and X is the matrix

    Involves double linear regression on insample returns {see attached notebook}
\end{sloppypar}



% Additional Resources


\section*{Additional Resources}
\begin{itemize}
    \item {Source \url{https://www.moneyscience.com/50-questions-for-quant-interviews/}}
    \sloppy
    \item \url{https://www.janestreet.com/static/pdfs/trading-interview.pdf?utm_source=web&utm_medium=pdf&utm_campaign=probability_markets_guide}
\end {itemize}
\end{document}
